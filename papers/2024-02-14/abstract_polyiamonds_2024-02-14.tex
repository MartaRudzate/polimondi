\documentclass[a4paper,12pt]{article}


\pagestyle{empty}
\renewcommand{\thefootnote}{}
\newtheorem{theorem}{\bf Theorem}
\newtheorem{lemma}      [theorem] {\bf Lemma}
\newtheorem{corollary}  [theorem] {\bf Corollary}
\newtheorem{proposition} [theorem] {\bf Proposition}
\newtheorem{definition} [theorem] {\bf Definition}
\newtheorem{remark}     [theorem] {\bf Remark}
\newtheorem{example}    [theorem] {\bf Example}
\newtheorem{problem}    [theorem] {\bf Problem}


\begin{document}


\begin{center}
	 {\bf \large Families of Perfect Polyiamonds as Formal Languages}
% Mums granta nav :(
%	\footnote{{\bf Acknowledgement}  The support of the grant ...... is kindly announced.}
\end{center}

\vspace{0.1cm}

\begin{center}\large {
APSĪTIS Kalvis$^1$
\  and  \
RUDZĀTE Marta$^2$}
\end{center}
\begin{center}
\it {
$^1$ University of Latvia, 
Raiņa bulvāris 19, 
Latvia \\
E-mail: {\rm kalvis.apsitis@gmail.com} \\
\vspace{1mm}
$^2$ % Institution of the second author (only if it is different)\\
%Mailing Address \\
%Country \\
E-mail: {\rm martarudzate@gmail.com }}
\end{center}
\vspace{0.2cm}

\noindent
\begin{definition}
A {\em polyiamond} is a simple polygon made of equilateral triangles. A {\em perfect $n$-polyiamond}\cite{APSITIS1} is a polyiamond whose sides are of lengths $n,n-1,\ldots,1$ in this order. 
A {\em perfect acute polyiamond} has all interior angles $60^{\circ}$ or $300^{\circ}$, a {\em perfect obtuse polyiamond} has all angles $120^{\circ}$ or $240^{\circ}$.
Perfect polyiamonds can be represented as strings over the alphabet 
$\Sigma = \{ \mathtt{a}, \mathtt{b}, \mathtt{c}, \mathtt{d}, \mathtt{e}, \mathtt{f} \}$ listing their side directions in decreasing length order. 
\end{definition}



\begin{proposition}
There is no infinite set of perfect polyiamonds that is a regular language over $\Sigma$. ({\em Hint:} Apply Pumping lemma.)
\end{proposition}



\begin{proposition}
There is an infinite set of perfect polyiamonds generated by (1) a context-free grammar; (2) a TAG grammar \cite{Joshi1997}. (Examples in \cite{RudzateApsitis2024}.)
\end{proposition}

\noindent
\begin{tabular}{ |l|l| } 
\hline
Perfect polyiamonds & Acute perfect polyiamonds \\ 
$S \rightarrow \mathtt{acec}P\mathtt{db}$, & 
$S \rightarrow \mathtt{acaeae}P\mathtt{eacacac}$, \\ 
$P \rightarrow \mathtt{ecea}P\mathtt{abcb} \,\mid\, \mathtt{eceaeafabcb}.$ & 
$P \rightarrow \mathtt{aececacece}P(\mathtt{caea})^5 \,\mid\, \mathtt{cecececa}.$\\ 
\hline
\end{tabular}



\begin{proposition}
There is a {\em parallel context-free language (PCL)}\cite{Siromoney1974} generating an infinite two-dimensional family of perfect polyiamonds.
% express this with PCL rules: {\tt \url{https://bit.ly/3UoJnkq}}.
\end{proposition}

%\begin{proposition}
%For $n = 27, 35, 171, 275, 1595$, a perfect acute $n$-polyiamond with the 
%maximum area among all perfect acute $n$-polyiamonds has a triangular shape and matches the regex $\mathtt{a}(\mathtt{ca})^{+}(\mathtt{ea})^{+}(\mathtt{ec})^{+}(\mathtt{ac})^{+}$. 
%\end{proposition}


\begin{thebibliography}{10}

\bibitem{APSITIS1}
%(TODO) Atsauce uz rakstu, kur pirmoreiz nodefinēti perfektie polimondi.
Buliņa, E. \textit{Maģiskie polimondi un to īpašības} [{\it Magic Polyiamonds and their Properties}], Master's thesis, University of Latvia. Available at: {\tt https://dspace.lu.lv/dspace/handle/7/38915}
\bibitem{Joshi1997}
Joshi, A.K., Schabes, Y.
{\it Tree-Adjoining Grammars.}
In "Handbook of Formal Languages" (Volume 3), Springer, (1997) 69--123.
\bibitem{RudzateApsitis2024}
Rudzāte, M., Apsītis, K.
{\it Notes for the 'Families of Perfect Polyiamonds'.} 
Available at: {\tt https://bit.ly/49mqmmL}
\bibitem{Siromoney1974}
Siromoney, R., Krithivasan, K.
{\it Parallel Context-Free Languages.}
Information and Control. {\bf 24} (1974) 155--162. 

\end{thebibliography}




\end{document}

